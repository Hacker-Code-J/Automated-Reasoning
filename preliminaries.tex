\begin{tcolorbox}[colframe=defcolor,title={\color{white}\bf Truth Function}]
	\begin{definition}
		Let $\B=\set{\true,\false}$ be the \textit{boolean domain}. Let $k\in\N$. A mapping \[
		f:\B^k\to\B
		\] is called a \textbf{truth function}.
	\end{definition}
\end{tcolorbox}
\begin{remark}[Truth Functions of Connectives]
	The logical connectives are assumed to be \textbf{truth-functional}.
	Hence, they are represented by certain \textbf{truth functions}.
	\begin{itemize}
		\item[] \textbf{Logical Negation} The \textit{logical not connective} defines the truth function $f^{\lnot}$
		as follows:
		\begin{align*}
			f^{\lnot}(\false)&=\true\\
			f^{\lnot}(\true)&=\false
		\end{align*}
		\item[] \textbf{Logical Conjunction} The \textit{conjunction connective} defines the truth function $f^{\land}$
		as follows:
		\begin{align*}
			f^{\land}(\true,\true)&=\true\\
			f^{\land}(\true,\false)&=\false\\
			f^{\land}(\false,\true)&=\false\\
			f^{\land}(\false,\false)&=\false
		\end{align*}
		\item[] \textbf{Logical Disjunction} The \textit{disjunction connective} defines the truth function $f^{\lor}$
		as follows:
		\begin{align*}
			f^{\lor}(\true,\true)&=\true\\
			f^{\lor}(\true,\false)&=\true\\
			f^{\lor}(\false,\true)&=\true\\
			f^{\lor}(\false,\false)&=\false
		\end{align*}
	\end{itemize}
\end{remark}

\begin{tcolorbox}[colframe=procolor,title={\color{white}\bf Count of Truth Functions}]
\begin{proposition}
There are $2^{(2^k)}$ distinct truth functions on $k$ variables.
\end{proposition}
\end{tcolorbox}
\begin{proof}
Let $f:\B^k\to\B$ be a truth function for $k\in\N$. Then
\begin{itemize}
	\item[] (Cardinality of Cartesian Product of Finite Sets)
	\begin{align*}
		\#(\B^k)&=\#(\overbrace{\B\times\cdots\times\B}^{\text{$k$ times}})\\
		&=\overbrace{\#\B\#\B\cdots\#\B}^{\text{$k$ times}}\\
		&=\overbrace{2\cdot 2\cdots2}^{\text{$k$ times}}\\
		&=2^k.
	\end{align*}
	\item[] (Cardinality of Set of All Mappings.)
	\begin{align*}
	\#(T^S):=\set{f\subseteq S\times T:\text{$f$ is a mapping}}=(\# T)^{(\# S)}
	\implies \#(\B^{(\B^k)})=2^{(2^k)}
	\end{align*}
\end{itemize}
\end{proof}

\begin{tcolorbox}[colframe=corcolor,title={\color{white}\bf Unary Truth Functions}]
	\begin{corollary}
		There are 4 distinct unary truth functions:
		\begin{itemize}
			\item The \textit{constant function} $f(p)=\false$
			\item The \textit{constant function} $f(p)=\true$
			\item The \textit{identity function} $f(p)=p$
			\item The \textit{logical not function} $f(p)=\lnot p$
		\end{itemize}
	\end{corollary}
\end{tcolorbox}
\begin{proof}
From Count of Truth Functions there are $2^{(2^1)}=4$
distinct truth functions on 1 variable. These can be depicted in a truth table as follows:
\begin{table}[h!]\centering
\begin{tabular}{|c|cccc|}
\hline
$p$ & $\circ_1$ & $\circ_2$ & $\circ_3$ & $\circ_4$ \\
\hline
$\true$ & $\true$ & $\true$ & $\false$ & $\false$ \\
$\false$ & $\false$ & $\true$ & $\false$ & $\false$ \\
\hline
\end{tabular}
\end{table}
\begin{itemize}
	\item[$\circ_1$:] Whether $p=\true$ or $p=\false$, $\circ_1(p)=\true$. Thus $\circ_1$ is the \textit{constant function} $\circ_1(p)=\true$.
	\item[$\circ_2$:] We have \begin{enumerate}[(1)]
		\item $p=\true\implies\circ_2(p)=\true$
		\item $p=\false\implies\circ_2(p)=\false$
	\end{enumerate} Thus $\circ_2$ is the \textit{identity function} $\circ_2(p)=p$.
	\item[$\circ_3$:] We have \begin{enumerate}[(1)]
		\item $p=\true\implies\circ_3(p)=\false$
		\item $p=\false\implies\circ_3(p)=\true$
	\end{enumerate} Thus $\circ_3$ is the \textit{logical not function} $\circ_3(p)=\lnot p$.
	\item[$\circ_4$:] Whether $p=\true$ or $p=\false$, $\circ_4(p)=\false$. Thus $\circ_1$ is the \textit{constant function} $\circ_4(p)=\false$.
\end{itemize}
\end{proof}

\begin{tcolorbox}[colframe=corcolor,title={\color{white}\bf Binary Truth Functions}]
\begin{corollary}
There are 16 distinct unary truth functions:
\begin{itemize}
\item Two \textit{constant operations}:
	\begin{itemize}
		\item $f_{\true}(p,q)=\true$
		\item $f_{\false}(p,q)=\false$
	\end{itemize}
\item Two \textit{projections}:
	\begin{itemize}
		\item $\mathsf{Proj}_1(p,q)=p$
		\item $\mathsf{Proj}_2(p,q)=q$
	\end{itemize} 
\item Two \textit{negated projections}:
	\begin{itemize}
	\item $\overline{\mathsf{Proj}_1}(p,q)=\lnot p$
	\item $\overline{\mathsf{Proj}_2}(p,q)=\lnot q$
	\end{itemize} 
\item The \textit{conjunction}: $p\land q$
\item The \textit{disjunction}: $p\lor q$
\item Two \textit{conditionals}:
	\begin{itemize}
		\item $p\implies q$
		\item $q\implies p$
	\end{itemize}
\item The \textit{biconditional} (iff): $p\iff q$
\item The \textit{exclusive or} (xor): $\lnot(p\iff q)$
\item Two \textit{negated conditionals}:
	\begin{itemize}
		\item $\lnot(p\implies q)$
		\item $\lnot(q\implies p)$
	\end{itemize}
\item The \textit{NAND} $p\uparrow q$
\item The \textit{NOR} $p\downarrow q$
\end{itemize}
\end{corollary}
\end{tcolorbox}
\begin{proof}
	From Count of Truth Functions there are $2^{(2^2)}=16$
	distinct truth functions on 2 variable. These can be depicted in a truth table as follows:
	\begin{table}[h!]\centering
		\begin{tabular}{|c|cccc|}
			\hline
			$p$ & $\circ_1$ & $\circ_2$ & $\circ_3$ & $\circ_4$ \\
			\hline
			$\true$ & $\true$ & $\true$ & $\false$ & $\false$ \\
			$\false$ & $\false$ & $\true$ & $\false$ & $\false$ \\
			\hline
		\end{tabular}
	\end{table}
\end{proof}